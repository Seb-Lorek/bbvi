% --- Coverpage ---%
\thispagestyle{empty}
\topskip0pt

% Vertical centering (start)
\vspace*{\fill}

% University logo
\begin{figure}[htbp]
  \centering
  \includegraphics[width=0.5\textwidth]{logo_uni_goe.pdf}
\end{figure}

\bigskip

\begin{center}

% Title
\rule{\linewidth}{0.5mm}

{\Large \textbf{A Stochastic Variational Inference Approach for Semiparametric Regression}}

\rule{\linewidth}{0.5mm} \\

\end{center}

\bigskip

% Author
\begin{center}

by \\

\bigskip

Lorek, Sebastian \\
\texttt{sebastian.lorek@stud.uni-goettingen.de}

\end{center}

\bigskip

% Degree
\begin{center}

A Master's Thesis \\
Submitted to the Chair of Statistics \\
Faculty of Business and Economics, University of Göttingen \\
In Fulfillment of the Requirements \\
For a Master's Degree in Applied Statistics\\
First Supervisor: Prof. Dr. Kneib \\
Second Supervisor: Mr. Callegher \\
November, 2023 \\

\end{center}


% Vertical centering (end)
\vspace*{\fill}

% For TOC etc. already
\clearpage
\pagenumbering{Roman}

% --- Abstract --- %

\section*{Abstract}

The thesis first provides an introduction into Variational Inference as an alternative
inference tool in Bayesian inference problemes, which allows for approximation of the posterior distribution.
Moreover are the properties of stochastic variational inference for Bayesian GAMLSS models analyized.
Showing that using a black-box variational inference inference algorithm results in
consistent parameter estimates, using the mean field family, for a variety of common models.
Furthermore is variational inference compared to common MCMC sampling algorithms. The algorithm is implemented in \verb|python|
using \verb|JAX| as a numerical computing library and for automatic differentiation.
Moreover are the models constructed using graph theory via the \verb|networkx| package.

\clearpage

% --- TOC --- %
\tableofcontents
\clearpage

% --- LOF --- %
\listoffigures

% --- LOT --- %
\listoftables

% --- List of abbreviations --- %
\section*{List of Abbreviations}

\begin{tabular}{@{} l @{\hskip 1in} l}
  VI & Variational inference \\
  SVI & Stochastic variational inference \\
  MCMC & Markov chain Monte Carlo \\
  KL & Kullback-Leibler \\
  GAMLSS & Generalized additive models for location, scale and shape \\
  CAVI & Coordinate ascent variational inference \\
\end{tabular}
