% --- Coverpage ---%
\thispagestyle{empty}
\topskip0pt

% Vertical centering (start)
\vspace*{\fill}

% University logo
\begin{figure}[htbp]
  \centering
  \includegraphics[width=0.5\textwidth]{img/logo_uni_goe.pdf}
\end{figure}

\bigskip

\begin{center}

% Title
\rule{\linewidth}{0.5mm}

{\Large \textbf{A Stochastic Variational Inference Approach for Semiparametric
                Distributional Regression}}

\rule{\linewidth}{0.5mm} \\

\end{center}

\bigskip

% Author
\begin{center}

by \\

\bigskip

Lorek, Sebastian \\
Matriculation number: 20311558

\end{center}

\bigskip

% Degree
\begin{center}

A Master's Thesis \\
Submitted to the Chair of Statistics \\
Faculty of Business and Economics, University of Göttingen \\
In Fulfillment of the Requirements \\
For a Master's Degree in Applied Statistics\\
First Supervisor: Prof. Dr. Thomas Kneib \\
Second Supervisor: Gianmarco Callegher \\
December, 2023 \\

\end{center}


% Vertical centering (end)
\vspace*{\fill}

% For TOC etc. already
\clearpage
\pagenumbering{Roman}

% --- Abstract --- %

\section*{Abstract}

The thesis explores variational inference as a viable alternative tool for 
tackling Bayesian inference problems, allowing for the approximation of 
posterior distributions through optimization. We study "black box" variational inference, 
a method enabling inference in numerous probabilistic models without requiring model-specific derivations.
The "black box" variational inference algorithm has been implemented in Python 
within a software package consisting of a model building and inference library 
that leverage the framework of probabilistic graphical models.
A simulation study shows that the implemented variational inference algorithm results in 
consistent estimates when considering the posterior means of the model parameters. 
Furthermore, the thesis conducts a simulation study to analyze the posterior distributions 
of Markov chain Monte Carlo and "black box" variational inference, 
revealing that variational inference yields rather similar posterior distributions.
Finally variational inference indicates, the ability to 
approximate the posterior well even after a low number of optimization iterations. 
This is in contrast to Markov Chain Monte Carlo methods which usually require a 
large number of sampling iterations to obtain a good approximation of the posterior.

\clearpage

% --- TOC --- %
\setcounter{tocdepth}{2}
\tableofcontents
\clearpage

% --- LOF --- %
\listoffigures

% --- LOA --- %
\listofalgorithms 

% --- LOT --- %
\listoftables

% --- List of abbreviations --- %
\section*{List of Abbreviations}

\begin{tabular}{@{} l @{\hskip 1in} l}
  VI & Variational inference \\
  MCMC & Markov chain Monte Carlo \\
  KL & Kullback-Leibler \\
  BBVI & "Black box" variational inference \\
  GAMLSS & Generalized additive models for location, scale and shape \\
  DAG & Directed acyclic graph \\ 
  SVI & Stochastic variational inference \\
  CAVI & Coordinate ascent variational inference \\
  SGD & Stochastic gradient descent \\
  API & Application programming interface \\
  JIT & Just-in-time compilation \\
  CPU & Central processing unit \\
  GPU & Graphical processing unit \\
  TPU & Tensor processing unit \\
  RS & Rigby and Stasinopoulos \\
  CG & Cole and Green \\
  EmpSE & Empirical standard error \\
  SE & Standard error \\
  DGP & Data generating process \\
  IWLS & Iterated weighted least squares \\
  NUTS & No U-Turn sampler \\
\end{tabular}
