% For bullet list: use square or circle
\setbeamertemplate{itemize items}[square]

% Define your colors like this:
\usepackage{xcolor}
\definecolor{darkblue}{RGB}{0,101,141}
\definecolor{mblue}{RGB}{0,147,199}
\definecolor{lightblue}{RGB}{132,191,234}
\definecolor{grey60}{RGB}{135,135,134}

% Use your colors in the presentation like this:
\setbeamercolor{titlelike}{fg=darkblue}
\setbeamercolor{itemize item}{fg=darkblue}
\setbeamercolor{itemize subitem}{fg=darkblue}
\setbeamercolor{enumerate item}{fg=darkblue}

% change colors of the reference entries
\setbeamercolor{bibliography item}{fg=darkblue}
\setbeamercolor{bibliography entry author}{fg=darkblue}
\setbeamercolor{bibliography entry location}{fg=grey60}
\setbeamercolor{bibliography entry note}{fg=grey60}

% For nice boxse (e.g. for important statements).
\usepackage{tcolorbox}

% Slide numbers and logo manually
\usepackage{tabularx}
\setbeamertemplate{footline}{
    \color{fg}
    \begin{tabularx}{\textwidth}{XX}
    \insertframenumber/\inserttotalframenumber\hfill\phantom{.} &
	\hfill\includegraphics[width=3cm, height=1cm]{assets/logo_uni_goe.pdf}
    \\
    \end{tabularx}
}

% emojies 
\usepackage{hwemoji}

% tikZ
\usepackage{tikz}
\usetikzlibrary{bayesnet}

% algorithm2e
\usepackage{algorithm2e}

% Use threepartstable 
\usepackage[flushleft]{threeparttable}

% --- Mathematics --- %

% vectors
\newcommand{\zerovec}{\mathbf{0}}
\newcommand{\onevec}{\mathbf{1}}
\newcommand{\yvec}{\mathbf{y}}
\newcommand{\xvec}{\mathbf{x}}
\newcommand{\gvec}{\mathbf{g}}
\newcommand{\fvec}{\mathbf{f}}
\newcommand{\zvec}{\mathbf{z}}
\newcommand{\svec}{\mathbf{s}}


% matrices
\newcommand{\X}{\mathbf{X}}
\newcommand{\I}{\mathbf{I}}
\newcommand{\Lbold}{\mathbf{L}}
\newcommand{\J}{\mathbf{J}}
\newcommand{\Z}{\mathbf{Z}}
\newcommand{\K}{\mathbf{K}}
\newcommand{\D}{\mathbf{D}}
\newcommand{\T}{\mathbf{T}}

% Greek vectors/matrices
\newcommand{\thetavec}{\boldsymbol{\theta}}
\newcommand{\betavec}{\boldsymbol{\beta}}
\newcommand{\phivec}{\boldsymbol{\phi}}
\newcommand{\muvec}{\boldsymbol{\mu}}
\newcommand{\epsilonvec}{\boldsymbol{\epsilon}}
\newcommand{\varphivec}{\boldsymbol{\varphi}}
\newcommand{\gammavec}{\boldsymbol{\gamma}}
\newcommand{\etavec}{\boldsymbol{\eta}}

\newcommand{\Sigmabold}{\boldsymbol{\Sigma}}
\newcommand{\Thetabold}{\boldsymbol{\Theta}}

% other stuff
\DeclareMathOperator*{\argmax}{arg\,max}
\DeclareMathOperator*{\argmin}{arg\,min}
\DeclareMathOperator{\vect}{vec}
\DeclareMathOperator{\diag}{diag}

\newcommand{\ind}{\overset{\text{ind.}}{\sim}}
